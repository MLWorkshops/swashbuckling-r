% Options for packages loaded elsewhere
\PassOptionsToPackage{unicode}{hyperref}
\PassOptionsToPackage{hyphens}{url}
\PassOptionsToPackage{dvipsnames,svgnames,x11names}{xcolor}
%
\documentclass[
  letterpaper,
  DIV=11,
  numbers=noendperiod]{scrartcl}

\usepackage{amsmath,amssymb}
\usepackage{iftex}
\ifPDFTeX
  \usepackage[T1]{fontenc}
  \usepackage[utf8]{inputenc}
  \usepackage{textcomp} % provide euro and other symbols
\else % if luatex or xetex
  \usepackage{unicode-math}
  \defaultfontfeatures{Scale=MatchLowercase}
  \defaultfontfeatures[\rmfamily]{Ligatures=TeX,Scale=1}
\fi
\usepackage{lmodern}
\ifPDFTeX\else  
    % xetex/luatex font selection
\fi
% Use upquote if available, for straight quotes in verbatim environments
\IfFileExists{upquote.sty}{\usepackage{upquote}}{}
\IfFileExists{microtype.sty}{% use microtype if available
  \usepackage[]{microtype}
  \UseMicrotypeSet[protrusion]{basicmath} % disable protrusion for tt fonts
}{}
\makeatletter
\@ifundefined{KOMAClassName}{% if non-KOMA class
  \IfFileExists{parskip.sty}{%
    \usepackage{parskip}
  }{% else
    \setlength{\parindent}{0pt}
    \setlength{\parskip}{6pt plus 2pt minus 1pt}}
}{% if KOMA class
  \KOMAoptions{parskip=half}}
\makeatother
\usepackage{xcolor}
\setlength{\emergencystretch}{3em} % prevent overfull lines
\setcounter{secnumdepth}{-\maxdimen} % remove section numbering
% Make \paragraph and \subparagraph free-standing
\makeatletter
\ifx\paragraph\undefined\else
  \let\oldparagraph\paragraph
  \renewcommand{\paragraph}{
    \@ifstar
      \xxxParagraphStar
      \xxxParagraphNoStar
  }
  \newcommand{\xxxParagraphStar}[1]{\oldparagraph*{#1}\mbox{}}
  \newcommand{\xxxParagraphNoStar}[1]{\oldparagraph{#1}\mbox{}}
\fi
\ifx\subparagraph\undefined\else
  \let\oldsubparagraph\subparagraph
  \renewcommand{\subparagraph}{
    \@ifstar
      \xxxSubParagraphStar
      \xxxSubParagraphNoStar
  }
  \newcommand{\xxxSubParagraphStar}[1]{\oldsubparagraph*{#1}\mbox{}}
  \newcommand{\xxxSubParagraphNoStar}[1]{\oldsubparagraph{#1}\mbox{}}
\fi
\makeatother

\usepackage{color}
\usepackage{fancyvrb}
\newcommand{\VerbBar}{|}
\newcommand{\VERB}{\Verb[commandchars=\\\{\}]}
\DefineVerbatimEnvironment{Highlighting}{Verbatim}{commandchars=\\\{\}}
% Add ',fontsize=\small' for more characters per line
\usepackage{framed}
\definecolor{shadecolor}{RGB}{241,243,245}
\newenvironment{Shaded}{\begin{snugshade}}{\end{snugshade}}
\newcommand{\AlertTok}[1]{\textcolor[rgb]{0.68,0.00,0.00}{#1}}
\newcommand{\AnnotationTok}[1]{\textcolor[rgb]{0.37,0.37,0.37}{#1}}
\newcommand{\AttributeTok}[1]{\textcolor[rgb]{0.40,0.45,0.13}{#1}}
\newcommand{\BaseNTok}[1]{\textcolor[rgb]{0.68,0.00,0.00}{#1}}
\newcommand{\BuiltInTok}[1]{\textcolor[rgb]{0.00,0.23,0.31}{#1}}
\newcommand{\CharTok}[1]{\textcolor[rgb]{0.13,0.47,0.30}{#1}}
\newcommand{\CommentTok}[1]{\textcolor[rgb]{0.37,0.37,0.37}{#1}}
\newcommand{\CommentVarTok}[1]{\textcolor[rgb]{0.37,0.37,0.37}{\textit{#1}}}
\newcommand{\ConstantTok}[1]{\textcolor[rgb]{0.56,0.35,0.01}{#1}}
\newcommand{\ControlFlowTok}[1]{\textcolor[rgb]{0.00,0.23,0.31}{\textbf{#1}}}
\newcommand{\DataTypeTok}[1]{\textcolor[rgb]{0.68,0.00,0.00}{#1}}
\newcommand{\DecValTok}[1]{\textcolor[rgb]{0.68,0.00,0.00}{#1}}
\newcommand{\DocumentationTok}[1]{\textcolor[rgb]{0.37,0.37,0.37}{\textit{#1}}}
\newcommand{\ErrorTok}[1]{\textcolor[rgb]{0.68,0.00,0.00}{#1}}
\newcommand{\ExtensionTok}[1]{\textcolor[rgb]{0.00,0.23,0.31}{#1}}
\newcommand{\FloatTok}[1]{\textcolor[rgb]{0.68,0.00,0.00}{#1}}
\newcommand{\FunctionTok}[1]{\textcolor[rgb]{0.28,0.35,0.67}{#1}}
\newcommand{\ImportTok}[1]{\textcolor[rgb]{0.00,0.46,0.62}{#1}}
\newcommand{\InformationTok}[1]{\textcolor[rgb]{0.37,0.37,0.37}{#1}}
\newcommand{\KeywordTok}[1]{\textcolor[rgb]{0.00,0.23,0.31}{\textbf{#1}}}
\newcommand{\NormalTok}[1]{\textcolor[rgb]{0.00,0.23,0.31}{#1}}
\newcommand{\OperatorTok}[1]{\textcolor[rgb]{0.37,0.37,0.37}{#1}}
\newcommand{\OtherTok}[1]{\textcolor[rgb]{0.00,0.23,0.31}{#1}}
\newcommand{\PreprocessorTok}[1]{\textcolor[rgb]{0.68,0.00,0.00}{#1}}
\newcommand{\RegionMarkerTok}[1]{\textcolor[rgb]{0.00,0.23,0.31}{#1}}
\newcommand{\SpecialCharTok}[1]{\textcolor[rgb]{0.37,0.37,0.37}{#1}}
\newcommand{\SpecialStringTok}[1]{\textcolor[rgb]{0.13,0.47,0.30}{#1}}
\newcommand{\StringTok}[1]{\textcolor[rgb]{0.13,0.47,0.30}{#1}}
\newcommand{\VariableTok}[1]{\textcolor[rgb]{0.07,0.07,0.07}{#1}}
\newcommand{\VerbatimStringTok}[1]{\textcolor[rgb]{0.13,0.47,0.30}{#1}}
\newcommand{\WarningTok}[1]{\textcolor[rgb]{0.37,0.37,0.37}{\textit{#1}}}

\providecommand{\tightlist}{%
  \setlength{\itemsep}{0pt}\setlength{\parskip}{0pt}}\usepackage{longtable,booktabs,array}
\usepackage{calc} % for calculating minipage widths
% Correct order of tables after \paragraph or \subparagraph
\usepackage{etoolbox}
\makeatletter
\patchcmd\longtable{\par}{\if@noskipsec\mbox{}\fi\par}{}{}
\makeatother
% Allow footnotes in longtable head/foot
\IfFileExists{footnotehyper.sty}{\usepackage{footnotehyper}}{\usepackage{footnote}}
\makesavenoteenv{longtable}
\usepackage{graphicx}
\makeatletter
\def\maxwidth{\ifdim\Gin@nat@width>\linewidth\linewidth\else\Gin@nat@width\fi}
\def\maxheight{\ifdim\Gin@nat@height>\textheight\textheight\else\Gin@nat@height\fi}
\makeatother
% Scale images if necessary, so that they will not overflow the page
% margins by default, and it is still possible to overwrite the defaults
% using explicit options in \includegraphics[width, height, ...]{}
\setkeys{Gin}{width=\maxwidth,height=\maxheight,keepaspectratio}
% Set default figure placement to htbp
\makeatletter
\def\fps@figure{htbp}
\makeatother

\KOMAoption{captions}{tableheading}
\makeatletter
\@ifpackageloaded{tcolorbox}{}{\usepackage[skins,breakable]{tcolorbox}}
\@ifpackageloaded{fontawesome5}{}{\usepackage{fontawesome5}}
\definecolor{quarto-callout-color}{HTML}{909090}
\definecolor{quarto-callout-note-color}{HTML}{0758E5}
\definecolor{quarto-callout-important-color}{HTML}{CC1914}
\definecolor{quarto-callout-warning-color}{HTML}{EB9113}
\definecolor{quarto-callout-tip-color}{HTML}{00A047}
\definecolor{quarto-callout-caution-color}{HTML}{FC5300}
\definecolor{quarto-callout-color-frame}{HTML}{acacac}
\definecolor{quarto-callout-note-color-frame}{HTML}{4582ec}
\definecolor{quarto-callout-important-color-frame}{HTML}{d9534f}
\definecolor{quarto-callout-warning-color-frame}{HTML}{f0ad4e}
\definecolor{quarto-callout-tip-color-frame}{HTML}{02b875}
\definecolor{quarto-callout-caution-color-frame}{HTML}{fd7e14}
\makeatother
\makeatletter
\@ifpackageloaded{caption}{}{\usepackage{caption}}
\AtBeginDocument{%
\ifdefined\contentsname
  \renewcommand*\contentsname{Table of contents}
\else
  \newcommand\contentsname{Table of contents}
\fi
\ifdefined\listfigurename
  \renewcommand*\listfigurename{List of Figures}
\else
  \newcommand\listfigurename{List of Figures}
\fi
\ifdefined\listtablename
  \renewcommand*\listtablename{List of Tables}
\else
  \newcommand\listtablename{List of Tables}
\fi
\ifdefined\figurename
  \renewcommand*\figurename{Figure}
\else
  \newcommand\figurename{Figure}
\fi
\ifdefined\tablename
  \renewcommand*\tablename{Table}
\else
  \newcommand\tablename{Table}
\fi
}
\@ifpackageloaded{float}{}{\usepackage{float}}
\floatstyle{ruled}
\@ifundefined{c@chapter}{\newfloat{codelisting}{h}{lop}}{\newfloat{codelisting}{h}{lop}[chapter]}
\floatname{codelisting}{Listing}
\newcommand*\listoflistings{\listof{codelisting}{List of Listings}}
\makeatother
\makeatletter
\makeatother
\makeatletter
\@ifpackageloaded{caption}{}{\usepackage{caption}}
\@ifpackageloaded{subcaption}{}{\usepackage{subcaption}}
\makeatother

\ifLuaTeX
  \usepackage{selnolig}  % disable illegal ligatures
\fi
\usepackage{bookmark}

\IfFileExists{xurl.sty}{\usepackage{xurl}}{} % add URL line breaks if available
\urlstyle{same} % disable monospaced font for URLs
\hypersetup{
  pdftitle={A Short Introduction to R},
  pdfauthor={Micheleen},
  colorlinks=true,
  linkcolor={blue},
  filecolor={Maroon},
  citecolor={Blue},
  urlcolor={Blue},
  pdfcreator={LaTeX via pandoc}}


\title{A Short Introduction to R}
\author{Micheleen}
\date{2024-09-15}

\begin{document}
\maketitle

\renewcommand*\contentsname{Table of contents}
{
\hypersetup{linkcolor=}
\setcounter{tocdepth}{3}
\tableofcontents
}

\subsection{Introduction}\label{introduction}

This is an introduction to R meant to help anyone new to the language
gain familiarity with data types, data structures, the apply functions
and reading data. It is not comprehensive, but should serve as a nice
reference moving forward.

This is a Quarto-based document. Quarto enables you to weave together
content and executable code into a finished document. To learn more
about Quarto see \url{https://quarto.org}.

When you click the \textbf{Render} button a document will be generated
that includes both content and the output of embedded code.

\subsection{Outline}\label{outline}

\begin{itemize}
\tightlist
\item
  Basics

  \begin{itemize}
  \tightlist
  \item
    Loading packages
  \item
    Working directory
  \item
    Environment
  \item
    Information on objects
  \end{itemize}
\item
  Data types

  \begin{itemize}
  \tightlist
  \item
    \texttt{numeric}
  \item
    \texttt{integer}
  \item
    \texttt{complex}
  \item
    \texttt{character}
  \item
    \texttt{logical}
  \end{itemize}
\item
  Data structures

  \begin{itemize}
  \tightlist
  \item
    \texttt{vector}
  \item
    \texttt{factor}
  \item
    \texttt{array} and \texttt{matrix}
  \item
    \texttt{cbind} and \texttt{rbind}
  \item
    \texttt{data.frame}
  \item
    \texttt{list}
  \end{itemize}
\item
  \texttt{apply} functions
\item
  Read a file
\end{itemize}

\subsection{Basics}\label{basics}

To load a package you can use
\texttt{library(\textless{}package\ name\textgreater{})}.

\begin{Shaded}
\begin{Highlighting}[]
\FunctionTok{library}\NormalTok{(ggplot2)}
\end{Highlighting}
\end{Shaded}

To check which directory you are in, use the function \texttt{getwd} for
``get working directory'' (you can also set the working directory with
\texttt{setwd(\textless{}some\ dir\ path\textgreater{})}).

\begin{Shaded}
\begin{Highlighting}[]
\FunctionTok{getwd}\NormalTok{()}
\end{Highlighting}
\end{Shaded}

\begin{verbatim}
[1] "/Users/kadm/Documents/GitHub/rladies/swashbuckling-r/intro_and_viz"
\end{verbatim}

List all variables in the current environment (or use the Environment
panel in RStudio).

\begin{Shaded}
\begin{Highlighting}[]
\NormalTok{x }\OtherTok{\textless{}{-}} \StringTok{"Arrr"}
\FunctionTok{ls}\NormalTok{()}
\end{Highlighting}
\end{Shaded}

\begin{verbatim}
[1] "x"
\end{verbatim}

Get structure and info about an object.

\begin{Shaded}
\begin{Highlighting}[]
\CommentTok{\# The iris dataset is part of base R}
\FunctionTok{str}\NormalTok{(iris)}
\end{Highlighting}
\end{Shaded}

\begin{verbatim}
'data.frame':   150 obs. of  5 variables:
 $ Sepal.Length: num  5.1 4.9 4.7 4.6 5 5.4 4.6 5 4.4 4.9 ...
 $ Sepal.Width : num  3.5 3 3.2 3.1 3.6 3.9 3.4 3.4 2.9 3.1 ...
 $ Petal.Length: num  1.4 1.4 1.3 1.5 1.4 1.7 1.4 1.5 1.4 1.5 ...
 $ Petal.Width : num  0.2 0.2 0.2 0.2 0.2 0.4 0.3 0.2 0.2 0.1 ...
 $ Species     : Factor w/ 3 levels "setosa","versicolor",..: 1 1 1 1 1 1 1 1 1 1 ...
\end{verbatim}

Get the class of an object.

\begin{Shaded}
\begin{Highlighting}[]
\FunctionTok{class}\NormalTok{(x)}
\end{Highlighting}
\end{Shaded}

\begin{verbatim}
[1] "character"
\end{verbatim}

\begin{Shaded}
\begin{Highlighting}[]
\FunctionTok{class}\NormalTok{(iris)}
\end{Highlighting}
\end{Shaded}

\begin{verbatim}
[1] "data.frame"
\end{verbatim}

Remove a variable.

\begin{Shaded}
\begin{Highlighting}[]
\FunctionTok{rm}\NormalTok{(x)}
\end{Highlighting}
\end{Shaded}

\subsection{Data Types}\label{data-types}

\begin{Shaded}
\begin{Highlighting}[]
\CommentTok{\# numeric}
\FloatTok{3.14}
\end{Highlighting}
\end{Shaded}

\begin{verbatim}
[1] 3.14
\end{verbatim}

\begin{Shaded}
\begin{Highlighting}[]
\CommentTok{\# integer (L forces R to not store any decimals)}
\DecValTok{2}\NormalTok{L}
\end{Highlighting}
\end{Shaded}

\begin{verbatim}
[1] 2
\end{verbatim}

\begin{Shaded}
\begin{Highlighting}[]
\CommentTok{\# complex}
\DecValTok{7} \SpecialCharTok{+} \DecValTok{2}\NormalTok{i}
\end{Highlighting}
\end{Shaded}

\begin{verbatim}
[1] 7+2i
\end{verbatim}

\begin{Shaded}
\begin{Highlighting}[]
\CommentTok{\# character or string}
\StringTok{"Ahoy world!"}
\end{Highlighting}
\end{Shaded}

\begin{verbatim}
[1] "Ahoy world!"
\end{verbatim}

\begin{Shaded}
\begin{Highlighting}[]
\CommentTok{\# logical or boolean}
\ConstantTok{TRUE}
\end{Highlighting}
\end{Shaded}

\begin{verbatim}
[1] TRUE
\end{verbatim}

If you don't know your data type, you can query it with
\texttt{class()}.

\begin{Shaded}
\begin{Highlighting}[]
\FunctionTok{class}\NormalTok{(}\FloatTok{3.14}\NormalTok{)}
\end{Highlighting}
\end{Shaded}

\begin{verbatim}
[1] "numeric"
\end{verbatim}

\begin{Shaded}
\begin{Highlighting}[]
\FunctionTok{class}\NormalTok{(}\StringTok{"Ahoy world!"}\NormalTok{)}
\end{Highlighting}
\end{Shaded}

\begin{verbatim}
[1] "character"
\end{verbatim}

\subsection{Data Structures}\label{data-structures}

\subsubsection{Vectors}\label{vectors}

Use \texttt{c()} to define a vector. Even the number \texttt{1} is a
vector (equivalent to \texttt{c(1)}). \texttt{c} stands for ``combine''.

\begin{Shaded}
\begin{Highlighting}[]
\NormalTok{v }\OtherTok{\textless{}{-}} \FunctionTok{c}\NormalTok{(}\DecValTok{4}\NormalTok{,}\DecValTok{2}\NormalTok{,}\DecValTok{3}\NormalTok{,}\DecValTok{8}\NormalTok{,}\DecValTok{2}\NormalTok{,}\DecValTok{2}\NormalTok{,}\DecValTok{5}\NormalTok{)}
\NormalTok{v}
\end{Highlighting}
\end{Shaded}

\begin{verbatim}
[1] 4 2 3 8 2 2 5
\end{verbatim}

To get more info on data and data structure, use \texttt{is()} instead
of \texttt{class()}

\begin{Shaded}
\begin{Highlighting}[]
\FunctionTok{class}\NormalTok{(v)}
\end{Highlighting}
\end{Shaded}

\begin{verbatim}
[1] "numeric"
\end{verbatim}

\begin{Shaded}
\begin{Highlighting}[]
\FunctionTok{is}\NormalTok{(v)}
\end{Highlighting}
\end{Shaded}

\begin{verbatim}
[1] "numeric" "vector" 
\end{verbatim}

Throw in \texttt{NA} values into a vector.

\begin{Shaded}
\begin{Highlighting}[]
\NormalTok{v }\OtherTok{\textless{}{-}} \FunctionTok{c}\NormalTok{(}\DecValTok{4}\NormalTok{,}\DecValTok{2}\NormalTok{,}\DecValTok{3}\NormalTok{,}\DecValTok{8}\NormalTok{,}\DecValTok{2}\NormalTok{,}\ConstantTok{NA}\NormalTok{,}\DecValTok{5}\NormalTok{)}
\NormalTok{v}
\end{Highlighting}
\end{Shaded}

\begin{verbatim}
[1]  4  2  3  8  2 NA  5
\end{verbatim}

Reassigning a value in our vector.

\begin{Shaded}
\begin{Highlighting}[]
\NormalTok{v[}\DecValTok{4}\NormalTok{] }\OtherTok{=} \DecValTok{10}
\NormalTok{v}
\end{Highlighting}
\end{Shaded}

\begin{verbatim}
[1]  4  2  3 10  2 NA  5
\end{verbatim}

\begin{tcolorbox}[enhanced jigsaw, colbacktitle=quarto-callout-warning-color!10!white, title=\textcolor{quarto-callout-warning-color}{\faExclamationTriangle}\hspace{0.5em}{Python users!}, breakable, toprule=.15mm, coltitle=black, opacityback=0, bottomtitle=1mm, leftrule=.75mm, colframe=quarto-callout-warning-color-frame, bottomrule=.15mm, toptitle=1mm, arc=.35mm, titlerule=0mm, rightrule=.15mm, left=2mm, opacitybacktitle=0.6, colback=white]

Note that R starts indexing at 1, not 0 like python. So the first number
is our vector is accessed with \texttt{v{[}1{]}} in R.

\end{tcolorbox}

Indexing with multiple values. The \texttt{:} operator pulls all
integers from one value to another.

\begin{Shaded}
\begin{Highlighting}[]
\NormalTok{v[}\DecValTok{1}\SpecialCharTok{:}\DecValTok{3}\NormalTok{]}
\end{Highlighting}
\end{Shaded}

\begin{verbatim}
[1] 4 2 3
\end{verbatim}

Indexing a vector with another vector.

\begin{Shaded}
\begin{Highlighting}[]
\NormalTok{v[}\SpecialCharTok{{-}}\FunctionTok{c}\NormalTok{(}\DecValTok{1}\NormalTok{,}\DecValTok{5}\NormalTok{,}\DecValTok{8}\NormalTok{)]}
\end{Highlighting}
\end{Shaded}

\begin{verbatim}
[1]  2  3 10 NA  5
\end{verbatim}

Using a logical operator on a vector.

\begin{Shaded}
\begin{Highlighting}[]
\NormalTok{v }\SpecialCharTok{\textless{}} \DecValTok{4}
\end{Highlighting}
\end{Shaded}

\begin{verbatim}
[1] FALSE  TRUE  TRUE FALSE  TRUE    NA FALSE
\end{verbatim}

Subsetting based on a logical vector.

\begin{Shaded}
\begin{Highlighting}[]
\NormalTok{v[v}\SpecialCharTok{\textless{}}\DecValTok{4}\NormalTok{]}
\end{Highlighting}
\end{Shaded}

\begin{verbatim}
[1]  2  3  2 NA
\end{verbatim}

Note that \texttt{NA} are retained when subsetting with a logical
vector. This is because operations on \texttt{NA} yield \texttt{NA}, not
\texttt{FALSE}.

\begin{Shaded}
\begin{Highlighting}[]
\ConstantTok{NA} \SpecialCharTok{\textless{}} \DecValTok{4}
\end{Highlighting}
\end{Shaded}

\begin{verbatim}
[1] NA
\end{verbatim}

A character vector.

\begin{Shaded}
\begin{Highlighting}[]
\NormalTok{v }\OtherTok{\textless{}{-}} \FunctionTok{c}\NormalTok{(}\StringTok{"aye"}\NormalTok{,}\StringTok{"nay"}\NormalTok{,}\StringTok{"nay"}\NormalTok{,}\StringTok{"aye"}\NormalTok{,}\StringTok{"aye"}\NormalTok{)}
\NormalTok{v}
\end{Highlighting}
\end{Shaded}

\begin{verbatim}
[1] "aye" "nay" "nay" "aye" "aye"
\end{verbatim}

\subsubsection{Factors}\label{factors}

Factors are how we can define groupings in data. For example, we can
define two groups ``happy pirate'' and ``sad pirate''. It's simple to
convert a character vector into a factor with the \texttt{factor}
function.

\begin{Shaded}
\begin{Highlighting}[]
\NormalTok{groupings }\OtherTok{=} \FunctionTok{c}\NormalTok{(}\StringTok{"happy\_pirate"}\NormalTok{,}
              \StringTok{"happy\_pirate"}\NormalTok{,}
              \StringTok{"sad\_pirate"}\NormalTok{,}
              \StringTok{"happy\_pirate"}\NormalTok{,}
              \StringTok{"sad\_pirate"}\NormalTok{,}
              \StringTok{"sad\_pirate"}\NormalTok{,}
              \StringTok{"happy\_pirate"}\NormalTok{)}

\NormalTok{pirate.factor }\OtherTok{=} \FunctionTok{factor}\NormalTok{(groupings)}

\NormalTok{pirate.factor}
\end{Highlighting}
\end{Shaded}

\begin{verbatim}
[1] happy_pirate happy_pirate sad_pirate   happy_pirate sad_pirate  
[6] sad_pirate   happy_pirate
Levels: happy_pirate sad_pirate
\end{verbatim}

Just get the levels.

\begin{Shaded}
\begin{Highlighting}[]
\FunctionTok{levels}\NormalTok{(pirate.factor)}
\end{Highlighting}
\end{Shaded}

\begin{verbatim}
[1] "happy_pirate" "sad_pirate"  
\end{verbatim}

Get the counts.

\begin{Shaded}
\begin{Highlighting}[]
\FunctionTok{table}\NormalTok{(pirate.factor)}
\end{Highlighting}
\end{Shaded}

\begin{verbatim}
pirate.factor
happy_pirate   sad_pirate 
           4            3 
\end{verbatim}

Factors automatically take their order as alphabetical. You can change
this by defining the levels of your factor.

\begin{Shaded}
\begin{Highlighting}[]
\NormalTok{pirate.factor }\OtherTok{=} \FunctionTok{factor}\NormalTok{(groupings, }\AttributeTok{levels=}\FunctionTok{c}\NormalTok{(}\StringTok{"sad\_pirate"}\NormalTok{, }\StringTok{"happy\_pirate"}\NormalTok{))}
\FunctionTok{levels}\NormalTok{(pirate.factor)}
\end{Highlighting}
\end{Shaded}

\begin{verbatim}
[1] "sad_pirate"   "happy_pirate"
\end{verbatim}

Create a dataset of ``coins''. Use the groupings to apply the
\texttt{mean} function with \texttt{tapply}. We will cover
\texttt{tapply} later. You can see now that factors are helpful when you
have distinct groups and you would like to apply a function based on
group assignments.

\begin{Shaded}
\begin{Highlighting}[]
\NormalTok{coins }\OtherTok{=} \FunctionTok{c}\NormalTok{(}\DecValTok{21}\NormalTok{,}\DecValTok{39}\NormalTok{,}\DecValTok{19}\NormalTok{,}\DecValTok{16}\NormalTok{,}\DecValTok{7}\NormalTok{,}\DecValTok{8}\NormalTok{,}\DecValTok{25}\NormalTok{)}

\FunctionTok{tapply}\NormalTok{(coins, pirate.factor, mean)}
\end{Highlighting}
\end{Shaded}

\begin{verbatim}
  sad_pirate happy_pirate 
    11.33333     25.25000 
\end{verbatim}

\subsubsection{Arrays and matrices}\label{arrays-and-matrices}

An array is an N-dimensional series of data entries of all the same
type. A matrix is a special case of an array that is 2-dimensional and
only contains numeric data.

\begin{Shaded}
\begin{Highlighting}[]
\CommentTok{\# 1D array}
\NormalTok{arr1 }\OtherTok{=} \FunctionTok{array}\NormalTok{(}\FunctionTok{c}\NormalTok{(}\DecValTok{1}\NormalTok{,}\DecValTok{2}\NormalTok{,}\DecValTok{3}\NormalTok{), }\AttributeTok{dim=}\FunctionTok{c}\NormalTok{(}\DecValTok{3}\NormalTok{))}
\NormalTok{arr1}
\end{Highlighting}
\end{Shaded}

\begin{verbatim}
[1] 1 2 3
\end{verbatim}

\begin{Shaded}
\begin{Highlighting}[]
\CommentTok{\# 2D array}
\NormalTok{arr2 }\OtherTok{=} \FunctionTok{array}\NormalTok{(}\FunctionTok{c}\NormalTok{(}\DecValTok{1}\NormalTok{,}\DecValTok{2}\NormalTok{,}\DecValTok{3}\NormalTok{,}\DecValTok{4}\NormalTok{,}\DecValTok{5}\NormalTok{,}\DecValTok{6}\NormalTok{,}\DecValTok{7}\NormalTok{,}\DecValTok{8}\NormalTok{), }\AttributeTok{dim=}\FunctionTok{c}\NormalTok{(}\DecValTok{2}\NormalTok{,}\DecValTok{4}\NormalTok{))}
\NormalTok{arr2}
\end{Highlighting}
\end{Shaded}

\begin{verbatim}
     [,1] [,2] [,3] [,4]
[1,]    1    3    5    7
[2,]    2    4    6    8
\end{verbatim}

\begin{Shaded}
\begin{Highlighting}[]
\CommentTok{\# The equivalent matrix for the 2D array above}
\NormalTok{mat }\OtherTok{=} \FunctionTok{matrix}\NormalTok{(}\FunctionTok{c}\NormalTok{(}\DecValTok{1}\NormalTok{,}\DecValTok{2}\NormalTok{,}\DecValTok{3}\NormalTok{,}\DecValTok{4}\NormalTok{,}\DecValTok{5}\NormalTok{,}\DecValTok{6}\NormalTok{,}\DecValTok{7}\NormalTok{,}\DecValTok{8}\NormalTok{),}
             \AttributeTok{nrow=}\DecValTok{2}\NormalTok{,}
             \AttributeTok{ncol=}\DecValTok{4}\NormalTok{)}
\NormalTok{mat}
\end{Highlighting}
\end{Shaded}

\begin{verbatim}
     [,1] [,2] [,3] [,4]
[1,]    1    3    5    7
[2,]    2    4    6    8
\end{verbatim}

\begin{Shaded}
\begin{Highlighting}[]
\CommentTok{\# A third way to create the same data structure}
\NormalTok{x }\OtherTok{=} \FunctionTok{c}\NormalTok{(}\DecValTok{1}\NormalTok{,}\DecValTok{2}\NormalTok{,}\DecValTok{3}\NormalTok{,}\DecValTok{4}\NormalTok{,}\DecValTok{5}\NormalTok{,}\DecValTok{6}\NormalTok{,}\DecValTok{7}\NormalTok{,}\DecValTok{8}\NormalTok{)}
\FunctionTok{dim}\NormalTok{(x) }\OtherTok{=} \FunctionTok{c}\NormalTok{(}\DecValTok{2}\NormalTok{,}\DecValTok{4}\NormalTok{)}
\NormalTok{x}
\end{Highlighting}
\end{Shaded}

\begin{verbatim}
     [,1] [,2] [,3] [,4]
[1,]    1    3    5    7
[2,]    2    4    6    8
\end{verbatim}

Indexing.

\begin{Shaded}
\begin{Highlighting}[]
\CommentTok{\# First row}
\NormalTok{mat[}\FunctionTok{c}\NormalTok{(}\DecValTok{1}\NormalTok{),]}
\end{Highlighting}
\end{Shaded}

\begin{verbatim}
[1] 1 3 5 7
\end{verbatim}

\begin{Shaded}
\begin{Highlighting}[]
\CommentTok{\# Second and fourth column}
\NormalTok{mat[,}\FunctionTok{c}\NormalTok{(}\DecValTok{2}\NormalTok{,}\DecValTok{4}\NormalTok{)]}
\end{Highlighting}
\end{Shaded}

\begin{verbatim}
     [,1] [,2]
[1,]    3    7
[2,]    4    8
\end{verbatim}

Indexing with a logical vector.

\begin{Shaded}
\begin{Highlighting}[]
\CommentTok{\# Second row}
\NormalTok{mat[}\FunctionTok{c}\NormalTok{(}\ConstantTok{FALSE}\NormalTok{,}\ConstantTok{TRUE}\NormalTok{),]}
\end{Highlighting}
\end{Shaded}

\begin{verbatim}
[1] 2 4 6 8
\end{verbatim}

\paragraph{Binding rows and columns}\label{binding-rows-and-columns}

Use \texttt{cbind} to add a column of the same dimension as other
columns. Use \texttt{rbind} to do the same for rows.

\begin{Shaded}
\begin{Highlighting}[]
\CommentTok{\# A matrix}
\NormalTok{mat }\OtherTok{=} \FunctionTok{matrix}\NormalTok{(}\FunctionTok{c}\NormalTok{(}\DecValTok{1}\NormalTok{,}\DecValTok{2}\NormalTok{,}\DecValTok{3}\NormalTok{,}\DecValTok{4}\NormalTok{,}\DecValTok{5}\NormalTok{,}\DecValTok{6}\NormalTok{,}\DecValTok{7}\NormalTok{,}\DecValTok{8}\NormalTok{),}
             \AttributeTok{nrow=}\DecValTok{2}\NormalTok{,}
             \AttributeTok{ncol=}\DecValTok{4}\NormalTok{)}

\NormalTok{mat2 }\OtherTok{=} \FunctionTok{cbind}\NormalTok{(mat, }\FunctionTok{c}\NormalTok{(}\DecValTok{9}\NormalTok{,}\DecValTok{10}\NormalTok{))}
\NormalTok{mat2}
\end{Highlighting}
\end{Shaded}

\begin{verbatim}
     [,1] [,2] [,3] [,4] [,5]
[1,]    1    3    5    7    9
[2,]    2    4    6    8   10
\end{verbatim}

\begin{Shaded}
\begin{Highlighting}[]
\NormalTok{mat3 }\OtherTok{=} \FunctionTok{rbind}\NormalTok{(mat, }\FunctionTok{c}\NormalTok{(}\DecValTok{9}\NormalTok{,}\DecValTok{10}\NormalTok{,}\DecValTok{11}\NormalTok{,}\DecValTok{12}\NormalTok{))}
\NormalTok{mat3}
\end{Highlighting}
\end{Shaded}

\begin{verbatim}
     [,1] [,2] [,3] [,4]
[1,]    1    3    5    7
[2,]    2    4    6    8
[3,]    9   10   11   12
\end{verbatim}

\begin{tcolorbox}[enhanced jigsaw, colbacktitle=quarto-callout-warning-color!10!white, title=\textcolor{quarto-callout-warning-color}{\faExclamationTriangle}\hspace{0.5em}{Beware data order}, breakable, toprule=.15mm, coltitle=black, opacityback=0, bottomtitle=1mm, leftrule=.75mm, colframe=quarto-callout-warning-color-frame, bottomrule=.15mm, toptitle=1mm, arc=.35mm, titlerule=0mm, rightrule=.15mm, left=2mm, opacitybacktitle=0.6, colback=white]

Note that \texttt{cbind} and \texttt{rbind} do not check your data
order. If you have one matrix with columns A, B and one with B, A,
\texttt{rbind} will incorrectly paste the data from the first columns
together and label it A, even though the second matrix's first column is
data for B. See the \texttt{tidyverse} for binding functions that
automatically correct ordering based on row and column names,
\url{https://dplyr.tidyverse.org/reference/bind.html}

\end{tcolorbox}

\subsubsection{Dataframes}\label{dataframes}

Unlike a matrix, a data frame can contain multiple types of data. Let's
create some data.

\begin{Shaded}
\begin{Highlighting}[]
\CommentTok{\# Create vectors of values for our sharks}
\NormalTok{tailfin.length}\OtherTok{\textless{}{-}}\FunctionTok{c}\NormalTok{(}\FloatTok{3.5}\NormalTok{,}\FloatTok{3.0}\NormalTok{,}\FloatTok{3.2}\NormalTok{,}\FloatTok{3.2}\NormalTok{,}\FloatTok{3.3}\NormalTok{,}\FloatTok{2.7}\NormalTok{)}
\NormalTok{tailfin.width}\OtherTok{\textless{}{-}}\FunctionTok{c}\NormalTok{(}\FloatTok{5.1}\NormalTok{,}\FloatTok{4.9}\NormalTok{,}\FloatTok{7.0}\NormalTok{,}\FloatTok{6.4}\NormalTok{,}\FloatTok{6.3}\NormalTok{,}\FloatTok{5.8}\NormalTok{)}

\CommentTok{\# Create a vector of shark species (reasons not to walk the plank, matey!)}
\NormalTok{species }\OtherTok{=} \FunctionTok{c}\NormalTok{(}\StringTok{"GreatWhite"}\NormalTok{,}
            \StringTok{"GreatWhite"}\NormalTok{,}
            \StringTok{"Bull"}\NormalTok{,}
            \StringTok{"Bull"}\NormalTok{,}
            \StringTok{"Tiger"}\NormalTok{,}
            \StringTok{"Tiger"}\NormalTok{)}
\end{Highlighting}
\end{Shaded}

Create a data frame from our penguin data.

\begin{Shaded}
\begin{Highlighting}[]
\NormalTok{shark.df }\OtherTok{=} \FunctionTok{data.frame}\NormalTok{(tailfin.length, tailfin.width, species)}
\FunctionTok{table}\NormalTok{(shark.df}\SpecialCharTok{$}\NormalTok{species)}
\end{Highlighting}
\end{Shaded}

\begin{verbatim}

      Bull GreatWhite      Tiger 
         2          2          2 
\end{verbatim}

Slicing is the same as with a matrix.

\begin{Shaded}
\begin{Highlighting}[]
\CommentTok{\# Which dimension is this, matey?}
\NormalTok{shark.df[,}\DecValTok{2}\NormalTok{]}
\end{Highlighting}
\end{Shaded}

\begin{verbatim}
[1] 5.1 4.9 7.0 6.4 6.3 5.8
\end{verbatim}

Quick summary of the dataset.

\begin{Shaded}
\begin{Highlighting}[]
\FunctionTok{summary}\NormalTok{(shark.df)}
\end{Highlighting}
\end{Shaded}

\begin{verbatim}
 tailfin.length  tailfin.width     species         
 Min.   :2.700   Min.   :4.900   Length:6          
 1st Qu.:3.050   1st Qu.:5.275   Class :character  
 Median :3.200   Median :6.050   Mode  :character  
 Mean   :3.150   Mean   :5.917                     
 3rd Qu.:3.275   3rd Qu.:6.375                     
 Max.   :3.500   Max.   :7.000                     
\end{verbatim}

Assign some row names.

\begin{Shaded}
\begin{Highlighting}[]
\FunctionTok{rownames}\NormalTok{(shark.df) }\OtherTok{=} \FunctionTok{c}\NormalTok{(}\StringTok{"p1"}\NormalTok{,}\StringTok{"p2"}\NormalTok{,}\StringTok{"p3"}\NormalTok{,}\StringTok{"p4"}\NormalTok{,}\StringTok{"p5"}\NormalTok{,}\StringTok{"p6"}\NormalTok{)}
\NormalTok{shark.df}
\end{Highlighting}
\end{Shaded}

\begin{verbatim}
   tailfin.length tailfin.width    species
p1            3.5           5.1 GreatWhite
p2            3.0           4.9 GreatWhite
p3            3.2           7.0       Bull
p4            3.2           6.4       Bull
p5            3.3           6.3      Tiger
p6            2.7           5.8      Tiger
\end{verbatim}

\subsubsection{Lists}\label{lists}

A list is a collection of objects. It can contain vectors, matrices,
dataframes, and other objects of different lengths. It can be a great
way to collect together different information. Use double square
brackets, \texttt{{[}{[}{]}{]}}, to access elements or you can also use
the name of the element if it exists (using the \texttt{\$} to access).

\begin{Shaded}
\begin{Highlighting}[]
\CommentTok{\# Create a list}
\NormalTok{sharkfin\_list }\OtherTok{=} \FunctionTok{list}\NormalTok{(tailfin.width, tailfin.length, }\FunctionTok{c}\NormalTok{(}\StringTok{"GreatWhite"}\NormalTok{, }\StringTok{"Bull"}\NormalTok{, }\StringTok{"Tiger"}\NormalTok{))}
\NormalTok{sharkfin\_list}
\end{Highlighting}
\end{Shaded}

\begin{verbatim}
[[1]]
[1] 5.1 4.9 7.0 6.4 6.3 5.8

[[2]]
[1] 3.5 3.0 3.2 3.2 3.3 2.7

[[3]]
[1] "GreatWhite" "Bull"       "Tiger"     
\end{verbatim}

The following is the same list, but with names assigned to the elements.

\begin{Shaded}
\begin{Highlighting}[]
\CommentTok{\# Create a list}
\NormalTok{sharkfin\_list }\OtherTok{=} \FunctionTok{list}\NormalTok{(}\AttributeTok{width=}\NormalTok{tailfin.width,}
                     \AttributeTok{length=}\NormalTok{tailfin.length,}
                     \AttributeTok{species=}\FunctionTok{c}\NormalTok{(}\StringTok{"GreatWhite"}\NormalTok{, }\StringTok{"Bull"}\NormalTok{, }\StringTok{"Tiger"}\NormalTok{),}
                     \AttributeTok{numberOfBites=}\DecValTok{15}\NormalTok{)}
\CommentTok{\# Get list names}
\FunctionTok{names}\NormalTok{(sharkfin\_list)}
\end{Highlighting}
\end{Shaded}

\begin{verbatim}
[1] "width"         "length"        "species"       "numberOfBites"
\end{verbatim}

\begin{Shaded}
\begin{Highlighting}[]
\CommentTok{\# Access with [[]]}
\NormalTok{sharkfin\_list[[}\DecValTok{2}\NormalTok{]]}
\end{Highlighting}
\end{Shaded}

\begin{verbatim}
[1] 3.5 3.0 3.2 3.2 3.3 2.7
\end{verbatim}

\begin{Shaded}
\begin{Highlighting}[]
\CommentTok{\# Access with $}
\NormalTok{sharkfin\_list}\SpecialCharTok{$}\NormalTok{length}
\end{Highlighting}
\end{Shaded}

\begin{verbatim}
[1] 3.5 3.0 3.2 3.2 3.3 2.7
\end{verbatim}

\subsection{Apply function family}\label{apply-function-family}

The functions \texttt{apply()}, \texttt{tapply()} and \texttt{lapply()}
allow you to perform specified functions across array objects.

\begin{itemize}
\tightlist
\item
  \texttt{apply()} -- first provide the array then whether to apply the
  function by row (1), column (2) or both (c(1,2)), finally the
  function.
\item
  \texttt{tapply()} -- similar to apply but pass a factor vector instead
  of row or column
\item
  \texttt{lapply()} -- simply provide a list and the function to apply
  to each vector in the list. Result is a list.
\item
  \texttt{sapply()} -- same as \texttt{lapply()} but returns result in
  original data structure
\end{itemize}

\begin{Shaded}
\begin{Highlighting}[]
\CommentTok{\# A matrix}
\NormalTok{mat }\OtherTok{=} \FunctionTok{matrix}\NormalTok{(}\FunctionTok{sample}\NormalTok{(}\DecValTok{20}\SpecialCharTok{:}\DecValTok{160}\NormalTok{, }\DecValTok{20}\NormalTok{, }\AttributeTok{replace=}\NormalTok{T)}\SpecialCharTok{/}\DecValTok{10}\NormalTok{,}
             \AttributeTok{ncol=}\DecValTok{4}\NormalTok{,}
             \AttributeTok{nrow=}\DecValTok{5}\NormalTok{)}

\CommentTok{\# Let\textquotesingle{}s get the row means}
\FunctionTok{apply}\NormalTok{(mat, }\DecValTok{1}\NormalTok{, mean)}
\end{Highlighting}
\end{Shaded}

\begin{verbatim}
[1]  8.050  8.225 10.275 11.925 12.600
\end{verbatim}

\begin{Shaded}
\begin{Highlighting}[]
\CommentTok{\# Make into a data frame}
\NormalTok{df }\OtherTok{=} \FunctionTok{as.data.frame}\NormalTok{(mat)}
\NormalTok{groupings }\OtherTok{=} \FunctionTok{c}\NormalTok{(}\StringTok{"happy\_pirate"}\NormalTok{, }\StringTok{"happy\_pirate"}\NormalTok{, }\StringTok{"sad\_pirate"}\NormalTok{, }\StringTok{"sad\_pirate"}\NormalTok{)}

\CommentTok{\# Get the mean for the first row by the groups}
\FunctionTok{tapply}\NormalTok{(}\FunctionTok{as.numeric}\NormalTok{(df[}\DecValTok{1}\NormalTok{,]), groupings, mean)}
\end{Highlighting}
\end{Shaded}

\begin{verbatim}
happy_pirate   sad_pirate 
         6.5          9.6 
\end{verbatim}

I will leave \texttt{lapply()} and \texttt{sapply()} for you to learn
more about on your own.

\subsection{Reading in data}\label{reading-in-data}

Read a csv file.

\begin{Shaded}
\begin{Highlighting}[]
\CommentTok{\# Read data file with a header}
\NormalTok{pirate\_data }\OtherTok{=} \FunctionTok{read.csv}\NormalTok{(}\StringTok{"data/pirate\_data1.csv"}\NormalTok{,}
  \AttributeTok{header=}\ConstantTok{TRUE}\NormalTok{,}
  \AttributeTok{sep =} \StringTok{","}\NormalTok{)}
\end{Highlighting}
\end{Shaded}

What class of object is this?

\begin{Shaded}
\begin{Highlighting}[]
\FunctionTok{class}\NormalTok{(pirate\_data)}
\end{Highlighting}
\end{Shaded}

\begin{verbatim}
[1] "data.frame"
\end{verbatim}

\subsection{Help}\label{help}

Here are some ways to get help.

\begin{Shaded}
\begin{Highlighting}[]
\CommentTok{\# Help docs}
\NormalTok{?max}
\end{Highlighting}
\end{Shaded}

\begin{Shaded}
\begin{Highlighting}[]
\CommentTok{\# Any docs associated with the given input, here "summary"}
\NormalTok{??}\StringTok{"summary"}
\end{Highlighting}
\end{Shaded}





\end{document}
